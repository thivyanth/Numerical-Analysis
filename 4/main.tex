\documentclass{article}
\usepackage{amsmath}
\usepackage{amsfonts}

\begin{document}

\section*{Programming Assignment}

The $N$-point Gaussian quadrature is defined by the general formula
\begin{equation}
    \int_{a}^{b} K(x)f(x)dx \approx \sum_{i=1}^{N}w_if(x_i),
\end{equation}
where $K(x)$ is a weight function, $x_i$ are data points that are roots of a polynomial of degree $N$, and $w_i$ are the weights. A Gaussian quadrature function takes as input the number $N$, and in the output gives the roots $x_i$ and the weights $w_i$ corresponding to that quadrature. Write a computer program that will compute the following integrals using the mentioned quadrature. For all the integrals, plot the integral value as a function of $N$ to study the convergence behavior.

\begin{enumerate}
    \item[(a)] Compute
    \[
    I = \int_{-1}^{1} e^{-x^2} dx
    \]
    using the Gauss-Legendre quadrature. For what value of $N$ do you get a result converged up to six decimal places?
    
    \item[(b)] Compute
    \[
    I = \int_{0}^{\frac{\pi}{2}} \ln(1 + x) dx
    \]
    using the Gauss-Legendre quadrature. First, you should carry out a change of variables so that the limits of integration become $\mp 1$. The value of this integral correct to six decimal places is $I \approx 0.8565690$. For what value of $N$ do you get this result?
    
    \item[(c)] Compute
    \[
    I = \int_{0}^{\infty} e^{-x} \sin x dx
    \]
    using the Gauss-Laguerre quadrature. The exact value of this integral is $I = 0.5$; for what value of $N$ do you get a result accurate to six places of decimal?
    
    \item[(d)] Compute
    \[
    I = \int_{0}^{\infty} \frac{e^{-x} \sqrt{x}}{x + 4} dx
    \]
    using the Gauss-Laguerre quadrature. Your converged value should be close to $0.16776$.
    
    \item[(e)] Compute
    \[
    I = \int_{-\infty}^{\infty} e^{-x^2} \sin^2 x dx
    \]
    using the Gauss-Hermite quadrature. For $N = 8$, you should get the value $I \approx 0.560202$, accurate up to six places of decimal.
    
    \item[(f)] Compute
    \[
    I = \int_{-\infty}^{\infty} \frac{e^{-x^2}}{\sqrt{1 + x^2}} dx
    \]
    also using Gauss-Hermite quadrature. For $N = 12$, you should get $I \approx 0.15239$.
\end{enumerate}

\subsection*{Important Notes:}
You can use the following numpy functions for various Gaussian quadratures:
\begin{itemize}
    \item \texttt{polynomial.legendre.leggauss(deg)} for Gauss-Legendre quadrature
    \item \texttt{numpy.polynomial.laguerre.laggauss(deg)} for Gauss-Laguerre quadrature
    \item \texttt{polynomial.hermite.hermgauss(deg)} for Gauss-Hermite quadrature
\end{itemize}
For all the cases above, \texttt{deg} is nothing but $N$.

If in any computer program you need the numerical value of $\pi$, calculate it using the formula $\pi = 4\cdot\tan^{-1}(1.0)$. Note that in python, the function for computing $\tan^{-1}x$ is \texttt{math.atan(x)}.

\end{document}
